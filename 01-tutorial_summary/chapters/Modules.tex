\vspace{-1.3cm}
프로그램을 작성한 후에 인터프리터를 종료하면 전부 삭제되기 때문에, 코드를 어딘가에 저장해야 한다. 이때 .py로 끝나는 텍스트파일로 저장하고 파일을 실행하면 코드를 다시 입력하지 않아도 결과값을 얻을 수 있다.  이런 파이썬 코드들을 \uline{스크립트} 라고 한다.\\
\newline
모듈이란 이 중에서 함수, 변수 또는 클래스 등이 정의되어있는 .py 파일을 말한다. \\모듈의 이름은 저장된 파일의 이름과 같다. \\
파이썬에는 \_\_name\_\_ 이라는 전역변수가 있다. 모듈명은 이 전역변수로 제공되는데, 만약 그 파일 자체가 직접 실행되었다면 \_\_main\_\_ 이 출력되고, import된 파일이라면 파일의 이름이 출력된다.
\vspace{-0.4cm}
\begin{minted}
[
frame=lines,
framesep=3mm,
baselinestretch=1.3,
bgcolor=LightGray,
fontsize=\normalsize,
linenos
]
{python}
# my_module 파일 실행
print(__name__)  # __main__

# 다른 모듈에 import 됐을 때
import my_module
print(my_module.__name__)  # my_module
\end{minted}
모듈 안의 함수들 중 자주 사용하고싶은 것이 있다면 로컬 이름에 대입할 수 있다.
\vspace{-0.4cm}
\begin{minted}
[
frame=lines,
framesep=3mm,
baselinestretch=1.3,
bgcolor=LightGray,
fontsize=\normalsize,
linenos
]
{python}
func1 = my_module.func1
\end{minted}

\section{More on Modules}
모듈은 함수 뿐 아니라 실행 가능한 문장들도 포함할 수 있다. \\이 문장들은 모듈이 처음 import될 떄에 딱 한번만 실행되고, 함수들 또한 한 번 실행된 이후로는 더이상 실행되지 않고 정의 되기만 한다.\\
\newline
모듈 예시:
\vspace{-0.5cm}
\begin{minted}
[
frame=lines,
framesep=3mm,
baselinestretch=1.3,
bgcolor=LightGray,
fontsize=\normalsize,
linenos
]
{python}
# 피보나치 수열 모듈
def fib(n):
    """ n 이하의 피보나치 수 출력하기 """
    a, b = 0, 1
    while a < n:
        print(a,end=' ')
        a, b = b, a+b
    print()

def fib2(n):
    """ n 이하의 피보나치 수열 반환하기 """
    result = []
    a, b = 0, 1
    while a < n:
        result.append(a)
        a, b = b, a+b
    return result
\end{minted}
\vspace{-1cm}

각 모듈은 자신만의 네임스페이스를 가지고 있다. \\(네임스페이스: 이름\(\leftrightarrow\)객체를 저장하는 공간으로 딕셔너리와 비슷하게 생각할 수 있음). 모듈과 사용자의 전역 네임스페이스는 격리되어있기 때문에, 모듈 작성자는 사용자의 전역변수와 충돌할 위험에 대해 걱정하지 않아도 된다. \\
그럼에도 사용자가 모듈 내부의 전역변수의 값을 읽거나 변경하고자 한다면 \\'모듈이름.변수명' 형태로 접근 가능하다.  모듈 작성자의 입장에서는 의도치 않은 변경을 막고 싶은 변수의
변수명 앞에 \_ 를 붙여서 사용을 금지할 수 있다.\\

\newline
기본적으로는 모듈을 통째로 불러와서, 모듈명을 통해 내부 함수나 클래스에 접근한다. \\
하지만 그 대신에 모듈 내부의 특정 이름 자체를 사용자의 네임스페이스 안으로 가져올 수도 있다. 이 경우에는 해당 이름들만이 정의되고 모듈명은 정의되지 않는다.
\vspace{-0.4cm}
\begin{minted}
[
frame=lines,
framesep=3mm,
baselinestretch=1.3,
bgcolor=LightGray,
fontsize=\normalsize,
linenos
]
{python}
from fibo import fib, fib2
\end{minted}
모듈 안의 이름들 (언더바로 시작하는 이름 제외)을 모두 가져오는 import문도 있다.
\vspace{-0.4cm}
\begin{minted}
[
frame=lines,
framesep=3mm,
baselinestretch=1.3,
bgcolor=LightGray,
fontsize=\normalsize,
linenos
]
{python}
from fibo import *
\end{minted}
일반적으로 *을 임포트하는 것은 좋지 않다. 불러오려는 모듈과 현재 작업공간의 전역변수들이 충돌할 가능성이 높아지고, 가독성도 떨어지기 때문이다. \\
\newline
import문에서 모듈 이름 다음에 as가 올 경우 모듈명 대신 as뒤에 이어지는 이름을 사용할 수 있다.
\vspace{-0.4cm}
\begin{minted}
[
frame=lines,
framesep=3mm,
baselinestretch=1.3,
bgcolor=LightGray,
fontsize=\normalsize,
linenos
]
{python}
from fibo import fib as fibonacci
\end{minted}
\vspace{+0.4cm}
\subsection{Executing modules as scripts}
인터프리터에서 모듈을 바로 실행하면 다른 모듈에 import되었을 때와 똑같이 작동한다. 한 가지 다른 점은  \_\_name\_\_ 이 \_\_main\_\_ 으로 설정된다는 것이다. \\
이 모듈의 코드 끝에 아래의 코드를 붙이면,
\begin{minted}
[
frame=lines,
framesep=3mm,
baselinestretch=1.3,
bgcolor=LightGray,
fontsize=\normalsize,
linenos
]
{python}
if __name__ == "__main__":
    import sys
    fib(int(sys.argv[1]))
\end{minted}
모듈이 메인 파일로 열렸을 경우에는 스크립트로도 실행될 수 있다. \\
\newline
인터프리터에서는 python fibo.py <arguments> 형식으로 실행하면 되는데, \\이때 arguments(인수)로 넣은 값은 위 코드의 sys.argv[1]자리에 들어간다. \\만약 인수 자리에 여러 개의 값을 넣으면 argv[2], argv[3] ...에 차례로 들어가게 된다.
\begin{minted}
[
frame=lines,
framesep=3mm,
baselinestretch=1.3,
bgcolor=LightGray,
fontsize=\normalsize,
linenos
]
{python}
# fibo.py
def fib(n): ...
def fib2(n): ...
if __name__ == "__main__":
    import sys
    print(f'파일명: {sys.argv[0]}') # argv[0]: 파일 이름
    fib(int(sys.argv[1]))
    print(fib2(int(sys.argv[2])))

$ python fibo.py 10 20
>>> 파일명: fibo.py
    0 1 1 2 3 5 8
    [0, 1, 1, 2, 3, 5, 8, 13]
\end{minted}

\subsection{The Module Search Path}
어떤 모듈이 import되면, 인터프리터는 그 이름을 가진 내장 모듈을 먼저 찾는다.
만약 없다면 다음으로 sys.path라는 변수가 제공하는 디렉토리(목적지, 파일)들을 살펴본다.

sys.path의 디렉토리 리스트에는 \\
1. 현재 디렉터리(또는 입력 스크립트를 포함하는 디렉터리)
\(\rightarrow\) 현재 실행 중인 스크립트의 폴더에서 가장 먼저 모듈을 찾아보기 때문에\\
예를 들어, 만약 직접 만든 random.py라는 이름의 파일이 이 폴더 안에 있다면 진짜 random모듈 대신 내가 만든 엉뚱한 파일을 불러오게 된다.

2. PYTHONPATH (사용자 지정 경로)

3. 설치 의존적인 기본값
\(\rightarrow\) 파이썬 배포 시 포함되는 공식 기본 모듈(random, pathlib, ...) 또는 pip로 설치되는 외부 모듈(numpy, pandas, ...)


%모듈을 검색하는 경로에 직접 경로를 추가하려면 sys.path 리스트에 경로를 추가하면 됩니다. sys.path 리스트의 첫 번째 요소로 현재 디렉토리가 자동으로 포함되어 있습니다. 그 외에 추가로 경로를 추가해야 하는 경우에는 다음과 같이 수행할 수 있습니다:
\subsection{“Compiled” Python files}
컴파일이란?\\
작성된 소스 코드는 .py 파일로 저장되지만, 파이썬 인터프리터는 파일을 실행하기 전에 먼저 파이썬 바이트코드라는 형식으로 변환하는데 이 과정을 컴파일이라고 부른다. \\
\newline
.py파일이 한 번 실행되거나 import될 때 파이썬은 파일을 바이트코드(.pyc)파일로 변환해서 저장한다. 이렇게 .pyc 파일을 만들면 다음에 .py파일을 다시 컴파일할 필요가 없어 속도가 향상되기 때문이다. \\
.pyc 캐시파일들은 모두 \_\_pycache\_\_ 라는 폴더에 (모듈명).(버전).pyc 형태의 이름으로 저장된다. 예를 들어 파이썬 3.11에서 만들어진 spam.py라는 모듈이 있다면,  이 모듈의 캐시 파일 경로는 \_\_pycache\_\_/spam.cpython-311.pyc 가 된다.

파이썬에서는 소스 코드의 수정 날짜와 컴파일된 버전을 비교해서 최신 버전인지 항상 확인한다. 자동으로 처리되는 과정이기 때문에 직접 조작해줄 필요는 없다.\\
\newline
만약 모듈이 import되지 않고 명령창에서 직접 실행되는 경우에는 캐시를 확인하지 않고 매번 다시 컴파일하며, 결과 또한 저장되지 않는다. \\소스 코드 없이 캐시 파일만 존재하는 경우에도 캐시는 확인하지 않는다.

\section{Standard Modules}
sys 모듈의 ps1 와 ps2는 기본, 보조 프롬프트로 쓰이는 문자열을 정의하는 변수이다.
\begin{minted}
[
frame=lines,
framesep=3mm,
baselinestretch=1.3,
bgcolor=LightGray,
fontsize=\normalsize,
linenos
]
{python}
>>> import sys
>>> sys.ps1  # 현재 기본 프롬프트
'>>> '
>>> sys.ps2  # 현재 보조 프롬프트
'...: '
>>> sys.ps1 = '(^o^) '  # 기본 프롬프트 변경

(^o^) print('hello')
hello
(^o^) 5 * 3
15
(^o^) sys.exit()  # 인터프리터 끝내기
\end{minted}
변수 sys.path는 모듈 검색 경로 문자열들이 들어있는 리스트이다.
sys.path 리스트의 첫 번째 요소로 현재 디렉토리가 자동으로 포함되어 있고, 새로운 경로를 추가하려면 sys.path 리스트에 append해주면 된다.
\begin{minted}
[
frame=lines,
framesep=3mm,
baselinestretch=1.3,
bgcolor=LightGray,
fontsize=\normalsize,
linenos
]
{python}
import sys

sys.path.append("/path/to/module") # 경로 추가

# 경로 확인
print(sys.path)
>>> ['C:\\Users\\user\\anaconda3\\python312.zip',
     'C:\\Users\\user\\anaconda3\\DLLs',
     ...
     'C:\\Users\\user\\Desktop\\2-1\\프로그래밍\\실습',
     'C:\\Users\\user\\anaconda3\\Lib\\site-packages\\setuptools\\_vendor']
\end{minted}

\section{The dir() Function}
dir 함수는 모듈이 정해주는 이름들의 리스트를 돌려주는데,\\
크게 dir() 또는 dir(객체) 두 가지로 사용된다.\\
\newline
먼저, 인자가 없는 dir()은  현재 프로그램의 scope안에서 사용중인 변수들 \\즉 namespace 명단을 제공해준다.
\begin{minted}
[
frame=lines,
framesep=3mm,
baselinestretch=1.3,
bgcolor=LightGray,
fontsize=\normalsize,
linenos
]
{python}
x = 34
y = 'Python'
z = [1, 2, 3]

dir()
>>> ['__builtin__', '__builtins__', '__doc__',
    ... 'exit', 'get_ipython', 'open', 'quit',
    'x', 'y', 'z'] # 현재 사용중인 변수 x, y, z 확인 가능함
\end{minted}
객체명을 인자로 입력받게 되면, dir은 해당 객체의 attribute(변수, 메소드)명단을 돌려준다. 객체명으로는 변수, 함수, 모듈, 패키지, 데이터 타입 등이 올 수 있다.
\begin{minted}
[
frame=lines,
framesep=3mm,
baselinestretch=1.3,
bgcolor=LightGray,
fontsize=\normalsize,
linenos
]
{python}
# dir(데이터 타입 클래스) -> int, str, float, ...
dir(complex)
>>> ['__abs__', '__add__', '__bool__',
     '__class__', '__complex__', '__delattr__',
     ...
     '__truediv__', 'conjugate', 'imag', 'real']

# dir(객체 또는 변수명)
x = [1, 2, 3]
dir(x)
>>> ['__add__', '__class__',
     '__class_getitem__', '__contains__',
     ...
     'pop', 'remove', 'reverse', 'sort']

# dir(라이브러리 or 패키지 or 모듈명) -> math, random, matplotlib, ...
import datetime
dir(datetime)
>>> ['MAXYEAR', 'MINYEAR', 'UTC', '__all__',
     '__builtins__', '__cached__', '__doc__',
     ...
     'timedelta', 'timezone', 'tzinfo']
# (내장 함수들 또한 dir()이 아닌 builtin 모듈을 인수로 하는 dir함수로 확인해야함)
import builtins
dir(builtins)
>>> ['ArithmeticError', 'AssertionError',
     'AttributeError', 'BaseException',
     ...
     'tuple', 'type', 'vars', 'zip']
\end{minted}
\section{Packages}
패키지는 모듈을 모아 놓은 디렉토리를 말한다. “점으로 구분된 모듈 이름” 를 사용해서 파이썬의 모듈 네임스페이스를 구조화하는 방식이다. 예를 들어 패키지 A안에 있는 B라는 이름의 서브 모듈 이름은 A.B 가 된다.
다른 모듈의 작성자들이 서로의 전역변수 이름에 대해 걱정할 필요가 없는 것과 마찬가지로, 여러 개의 모듈이 포함된 패키지의 작성자들은 서로의 모듈 이름들을 걱정하지 않아도 된다.

패키지를 만들려면 우선 프로젝트 디렉토리(파일)을 만들어야 한다. 이 디렉토리의 이름이 패키지의 이름이 된다. 이제 모듈들을 추가하기 앞서  \_\_init\_\_.py 파일을 만들어야 한다.  \_\_init\_\_.py는 패키지를 초기화하는 데에 사용된다. 이 파일이 있어야 파이썬이 해당 디렉토리를 패키지로 인식하기 떄문에 초기화가 필요하지 않더라도 빈 파일을 생성해야 한다.
\begin{minted}
[
frame=lines,
framesep=3mm,
baselinestretch=1.3,
bgcolor=LightGray,
fontsize=\normalsize,
linenos
]
{python}
rootp/           # 최상위 패키지
    __init__.py  # rootp 패키지 초기화
    subp1/            # 서브패키지 1
        __init__.py   # 서브패키지 1 초기화
        mod1.py
        mod2.py
           ...
    subp2/           # 서브패키지 2
        __init__.py
        mod3.py
        mod4.py
           ...
    ...
\end{minted}
rootp, subp1, subp2는 패키지(디렉토리), 확장자가 .py인 파일들은 모듈이다.

패키지로부터 개별 모듈을 임포트할 수 있다.
\begin{minted}
[
frame=lines,
framesep=3mm,
baselinestretch=1.3,
bgcolor=LightGray,
fontsize=\normalsize,
linenos
]
{python}
# 1. 전체 이름 참조
import rootp.subp1.mod1
rootp.subp1.mod1.func(input, output, delay=0.7, atten=4)

# 2. 패키지 접두어 없이 모듈명 사용
from rootp.subp1 import mod1
mod1.func(input, output, delay=0.7, atten=4)

# 3. 원하는 함수 또는 변수 직접 임포트하기
from rootp.subp1.mod1 import func
func(input, output, delay=0.7, atten=4)
\end{minted}
중요한 규칙: \\
\textbf{import a.b.c} 형태의 임포트문에서 a, b는 항상 패키지여야 한다. 마지막에 오는 c는 패키지와 모듈 둘 다 가능하지만, 함수나 변수가 될 수는 없다. \\
반면 \textbf{from package import item}는 package자리에 모듈 또는 패키지, item 자리에 패키지, 모듈, 함수, 변수, 클래스 등등이 허용되어서 유연한 문법이고, 훨씬 간결하게 사용 가능하다. * 주의할 점은, item이 패키지의 init파일에 정의되지 않고, 서브패키지의 서브모듈 안에 포함된 함수나 변수라면 ImportError가 발생한다. 이를 예방하려면 package의 경로가 충분히 구체적이어야 한다. (item이 있는 곳까지 안내해주어야함)

\subsection{Importing * From a Package}
from package import * 라고만 쓰는 것은 패키지 파일 안에 있는 모듈들을 효과적으로 불러오기에는 보기보다 효율적이지 않다. 그래서 *을 그냥 임포트하는 대신, 패키지의 init파일 안에 임포트해야 할 모듈들의 목록(\_\_all\_\_)을 명시해주는 단계가 필요하다.\\
\newline
예를 들어 파일 \textit{rootp/subp2/\_\_init\_\_.py} 안에 \textit{ \_\_all\_\_ = ["mod3",  "mod4", "mod5"]}라는 코드를 포함시켜주면, from \textit{rootp.subp2} import * 가 위에서 지정해준 모듈들을 불러오게 된다.

모듈의 이름이 지역적으로 정의된 이름에 가려질 수 있다는 점도 주의해야 한다. \\
만약 rootp/subp2/\_\_init\_\_.py 파일에 mod5라는 이름의 함수가 추가된다면,\\
\_\_all\_\_ = ["mod3",  "mod4", "mod5"]에서 "mod5"는 더이상 모듈이 아닌 함수를 가리키게 되고  from \textit{rootp.subp2} import * 는 mod5라 모듈을 임포트하지 않는다.\\
\newline
\uline{\_\_all\_\_ 이 정의되지 않으면} from \textit{rootp.subp2} import * 는 subp2안의 서브 모듈들을  불러오지 않고 \uline{패키지 rootp.subp2와 그 패키지가 정의하는 이름들만 불러온다}. \\(패키지가 정의하는 이름들 = \_\_init\_\_.py 에 있는 초기화 코드에 명시된 이름들) \\init 파일 안에 정의되지 않았더라도, 이전에 임포트문을 통해 로드된 서브모듈이라면 import * 에 의해 현재 네임스페이스로 불려올 수 있다.
\begin{minted}
[
frame=lines,
framesep=3mm,
baselinestretch=1.3,
bgcolor=LightGray,
fontsize=\normalsize,
linenos
]
{python}
# rootp/subp2/__init__.py
pi = 3.14
def function(): ...  # 직접 정의
import rootp.subp2.mod3
# 변수나 함수와는 다르게, 모듈은 init안에 파일 형태로 존재하는 것만으로는
# * 을 통해 불려올 수 없음. init파일 안에서 import되어야 함.
#---------------------
# 실행 환경
import rootp.subp2.mod5
# from ~ inport * 보다 앞서서 임포트됨 -> 이미 subp2 안에 포함됨

from rootp.subp2 import *
''' --> pi, function, mod3, mod5 가 네임스페이스에 import됨 '''
\end{minted}
\subsection{Intra-package References}
패키지가 여러 개의 서브 패키지로 구성되어있을 경우 이웃 패키지의 서브모듈을\\ 가져올 수 있는데 절대 임포트, 상대 임포트의 두 가지 방법이 있다.\\
\vspace{-0.5cm}
\newline
my\_package 라는 가상의 패키지가 있다고 하자.
\vspace{-0.5cm}
\begin{minted}
[
frame=lines,
framesep=3mm,
baselinestretch=1.3,
bgcolor=LightGray,
fontsize=\normalsize,
linenos
]
{python}
my_package/
        __init__.py
        mod1.py
        mod2.py
        subp1/
            __init__.py
            mod_a.py
            mod_b.py
        subp2/
            __init__.py
            mod_c.py
        ...
\end{minted}
\uline{절대 임포트}\\
패키지 안에서 뿐만 아니라 어떤 모듈에서든 동일하게 작동하는 문법이다. \\
(바깥에서 참조할때도 사용 가능) \\
항상 최상위 패키지부터 경로가 시작된다
\begin{minted}
[
frame=lines,
framesep=3mm,
baselinestretch=1.3,
bgcolor=LightGray,
fontsize=\normalsize,
linenos
]
{python}
# mod_c에서 mod_a를 임포트할 때
from my_package.subp1 import mod_a

# 외부 패키지 python(예시) 에서 mod_a를 임포트할 때
from my_package.subp1 import mod_a  # 동일한 경로
\end{minted}
\uline{상대 임포트}\\
패키지 내부에서만 사용 가능하며, 현재 모듈의 위치를 기준으로 한다.
\begin{minted}
[
frame=lines,
framesep=3mm,
baselinestretch=1.3,
bgcolor=LightGray,
fontsize=\normalsize,
linenos
]
{python}
# 같은 디렉토리(my_package)의 형제 모듈
# mod2 에서 mod1 임포트
from . import mod1

# 상위 디렉토리(my_package)에 소속된 모듈
# mod_a 에서 mod1 임포트
from .. import mod1

# 같은 디렉토리의 서브패키지(subp1)에 속한 하위 모듈
# mod1 에서 mod_a 임포트
from .subp1 import mod_a

# 다른 서브패키지(subp2)의 모듈 (같은 레벨)
# mod_b 에서 mod_c 임포트
from .. subp2 import mod_c

# 다른 서브패키지(subp2)의 모듈(mod_c)의 함수
# mod_b 에서 mod_c 안의 함수 임포트
from .. subp2.mod_c import fx
\end{minted}
하지만 절대 임포트와 다르게 상대 임포트는 항상 가능하지는 않다.
\begin{tcolorbox}[colback=black!15!white,colframe=black!15!white]
\textit{상대 임포트를 위한 조건} \newline
어떤 모듈에서 상대 경로 임포트를 하려면, 그 모듈은 반드시 어떤 패키지(상위 \\디렉토리)안에서 실행중이어야 한다. \\
다시 말해, \uline{모듈의 \_\_name\_\_ 이 main이라면, 해당 모듈은 패키지의 일부로서 실행된 것이 아니기 때문에 상대 임포트 기능을 사용할 수 없다}.
\end{tcolorbox}
상대 경로는 \_\_name\_\_변수명을 통해 위치를 파악하고 경로를 계산하게 되는데, \\\_\_name\_\_이 "\_\_main\_\_"이 되면 상대 경로가 실행 파일의 실제 위치를  계산할 수 없다. \\
(상대 경로는 반드시 기준 위치를 알아야 한다. 기준 위치로부터 다른 파일의 위치를 추적하기 때문이다.)\\
따라서 모듈이 실제로 스크립트로 직접 실행된 경우, 또는 모듈이 패키지 모듈로서 실행되었는지를 파이썬이 명확하게 인식하지 못할 경우 오류가 발생할 수 있다.\\
( ImportError: attempted relative import with no known parent package)\\
따라서 안전한 임포트를 위해서는 많은 경우 절대임포트 방식이 더 적절하다.


