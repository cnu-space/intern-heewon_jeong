\vspace{-1cm}
\section{Fancier Output Formatting}

지금까지 사용했던 표현식을 포함하는 문장(expression statements)과 print()함수 \\이외에도 출력값 포맷팅을 위해 파이썬에서 제공하는 더 다양한 방법들이 있다 :\\
- formatted string literals\\
- 문자열의 str .format() 메소드\\
- 문자열 슬라이싱과 연결 연산을 사용하는 수동 포맷팅\\
- 주어진 너비에 문자열을 채우기 위한 메소드 (문자열 보간)\\

보기 좋게 정렬하는 것보다, 디버깅 과정을 위해 변수들을 빠르게 출력해야 한다면 repr() 또는 str() 함수를 사용할 수 있다.
str() 함수는 객체를 쉽게 읽고 이해할 수 있도록 사용자를 위한 형태의 값을 반환해준다.
반면 repr()함수는 인터프리터가 이해할 수 있는 '공식적인' 형태의 표현으로 값을 만들어 돌려준다. 만약 적절한 문법이 없다면 SyntaxError를 발생시키도록 되어있다.

사람이 사용하기 위한 특별한 표현이 따로 없는 객체들의 경우에는 str()도 repr()과 똑같은 값을 돌려준다. 숫자들 또는 리스트나 딕셔너리와 같은 구조들이 그 예시이다. 예외로 문자열은 두 함수에 의해 각각 다르게 표현된다.
\begin{minted}
[
frame=lines,
framesep=3mm,
baselinestretch=1.3,
bgcolor=LightGray,
fontsize=\normalsize,
linenos
]
{python}
hello = 'hello, world\n'
hello  # 'hello, world\n'
str(hello)  # 'hello, world\n'
repr(hello)  # "'hello, world\\n'"
print(str(hello))
print(repr(hello))
# hello, world
#                 ...(줄넘김 \n)
# 'hello, world\n'
## 문자열에서 repr()은 따옴표, 역슬래시를 추가함
## str()과 repr()을 print하면 각자의 방식으로 객체를 나타낸 문자열 안의 정보를 출력함

x = 10 * 3.25
y = '200'
r = 'x = ' + repr(x) + ', y = ' + repr(y)
s = 'x = ' + str(x) + ', y = ' + str(y)
print(r)  # x = 32.5, y = '200'
print(s)  # x = 32.5, y = 200
## 숫자 x는 repr(x)와 str(x) 값이 같고, 문자열인 y는 두 함수에서의 출력값이 다르다

x='1'+1  # TypeError
str('1')+repr(1)  # '11'
repr('1')+str(1)  # "'1'1"
## 두 함수는 모두 문자열을 돌려줌 (문자열 더하기 연산 가능)

i = [1,2,'a','b']
repr(i)  # "[1, '2', 3, 4, 5]"
str(i)  # "[1, '2', 3, 4, 5]"
## 리스트 객체는 두 함수에서 동일하게 표현됨
\end{minted}
\newpage
repr()의 사용 목적은 '객체를 문자열로 다시 생성하는 것'이라고 할 수 있다. \\
eval()함수와 함께 사용되는 예시를 보면 그 의미를 쉽게 이해할 수 있다. \\eval()은 repr()과는 반대로 문자열로 객체를 생성하는 함수이다. \\
다음과 같이 datetime 객체를 repr()로 생성한 문자열에 다시 eval() 함수를 실행하면 처음의 datetime 객체가 만들어져야 한다.
\begin{minted}
[
frame=lines,
framesep=3mm,
baselinestretch=1.3,
bgcolor=LightGray,
fontsize=\normalsize,
linenos
]
{python}
import datetime
today = datetime.date.today()

str(today)  # '2025-07-08'
repr(today)  # 'datetime.date(2025, 7, 8)'

eval(repr(today))  # datetime.date(2025, 7, 8)
eval(repr(today)) == today  # True
print(eval(repr(today)))  # 2025-07-08
# datetime.date(2025, 7, 8)이 의미하는 정보를 print하면 2025-07-08을 얻는다
\end{minted}

\subsection{Formatted String Literals}
포맷 문자열 리터럴, 또는 f-string은 f'문자열 {변수} 문자열' 형태로 문자열을 포맷하는 방법이다.
중괄호 사이에 변수 대신 여러 가지 표현식을 넣으면 훨씬 정교한 포맷팅이 가능하다.
\begin{minted}
[
frame=lines,
framesep=3mm,
baselinestretch=1.3,
bgcolor=LightGray,
fontsize=\normalsize,
linenos
]
{python}
s1 = 'left'
s2 = 'mid'
s3 = 'right'
result1 = f'|{s1:<10}|'  # 왼쪽 정렬
result2 = f'|{s2:^10}|'  # 가운데 정렬
result3 = f'|{s3:>10}|'  # 오른쪽 정렬
print(result1)
print(result2)
print(result3)

# 중괄호 출력하기
num = 1, 2, 3, 4, 5
result = f'squared = {{i**2 for i in {num}}}'
print(result)

# f-string과 딕셔너리
d = {'name': 'Bob bingi', 'gender': 'M', 'age': 5}
k = [key for key in d.keys()]
dog = f'{k[0]}: {d[k[0]]}\n{k[1]}: {d[k[1]]}\n{k[2]}: {d[k[2]]}'
print(dog)

# f-string과 리스트
a = [4,2,3,3,2,5,1,5,3]
print(f'<5 ~ {5 + len(a)}번 정답>')
for num, ans in enumerate(a, start = 5):
    print(f' 문제{num:>3d}: {ans}번')

# 문자열 안에서 함수 호출하기
a = complex(2, 3)
b = a.conjugate()
result = f'{a}의 크기\n= ({a}X{b})^(1/2)\n= {abs(a):.2f}'
print(result)

#시퀀스 슬라이싱하기
word = 'Python'
line1 = f'{word}의 첫 두 글자는 {word[:2]}입니다.'
line2 = f'{word}를 거꾸로 하면 {word[::-1]}입니다'
line3 = f'3회 반복: {','.join([word] * 3)}'

# 객체 치환하기
# f-string 안에서 객체를 사용하면 str(객체)의 결과가 자동으로 삽입된다.
from datetime import date
today = f"오늘은 {date.today()}입니다."
# 다른 수정자를 사용하면 포맷되기 전에 값을 변환할 수 있음.
#(!a'는 ascii(), '!s'는 str(), '!r'는 repr()를 적용한다)
today_r = f"오늘은 {date.today()!r}입니다."
print(today)
print(today_r)

# 디버깅
letter = 'a'
print(f'letter = {letter}')
letter = 'a'
print(f"{letter =}")
letter = 'a'
print(f"{letter = !s}")
\end{minted}

\subsection{The String format() Method}
문자열의 str.format() 메서드는 기본적으로 '문자열 {포맷 필드}문자열'.format(변수) \\형태로 쓰이는 메소드이다.\\
중괄호 안의 문자들(포맷 필드)은 .format()을 통해 전달된 객체들로 치환된다.\\
먼저, 중괄호 안에 0, 1, 2, ... 등의 숫자를 넣으면, 그 숫자들은 전달된 객체들의 위치를 가리킨다. 문자열.format('a',1,'hello')에서 'a'를 가리키려면 중괄호 안에 0을, 'hello'를 가리키려면 2를 사용하는 식이다.\\
만약 객체들이 키워드 인자 형태로 전달된다면 중괄호 안에 숫자 대신 인자의 이름을 넣는다. 한 줄의 문자열 안에서 위치와 키워드 인자들을 혼합해서 사용해도 된다.
\begin{minted}
[
frame=lines,
framesep=3mm,
baselinestretch=1.3,
bgcolor=LightGray,
fontsize=\normalsize,
linenos
]
{python}
print('{0}는 {1}와 {2}를 좋아한다'.format('로라', '메밀국수', '임금님 놀이'))
>>> 로라는 메밀국수와 임금님 놀이를 좋아한다

print('This {food} is {adjective}'.format(food = 'pizza', adjective = 'salty'))
>>> This pizza is salty

print('{sub1}: {1}점 / {sub2}: {1}점 / {sub3}: {0}점'.format(
      80, 90, sub1 = '국어', sub2 = '수학', sub3 = '영어'))
>>> 국어: 90점 / 수학: 90점 / 영어: 80점
\end{minted}
많은 수의 변수들을 사용하는 포맷 문자열을 만들 때에는 변수들을 위치 대신에 이름으로 지정해주는 것이 좋은데, 간단하게 딕셔너리를 format()으로 전달해줄 수 있다.
\begin{minted}
[
frame=lines,
framesep=3mm,
baselinestretch=1.3,
bgcolor=LightGray,
fontsize=\normalsize,
linenos
]
{python}
# 대괄호 안에 키를 넣어 포맷할 값에 액세스할 수 있다
midterm = {'Alice': 70, 'Bob': 80, 'James': 'absence'}
print('Alice: {0[Alice]:d} | Bob: {0[Bob]:d} | James: {0[James]:s}'
      .format(midterm))
>>> Alice: 70 | Bob: 80 | James: absence

# 또는 같은 딕셔너리를 **를 이용해 키워드 인자로 전달할 수도 있다
print('Alice: {Alice:d} | Bob: {Bob:d} | James: {James:s}'.format(**midterm))
>>> Alice: 70 | Bob: 80 | James: absence
# **midterm 키워드 인자들은 {Alice}자리에 70, {Bob}자리에 80, ...
# 으로 입력되어 포맷팅된다

--------------------------
# 모든 지역 변수들을 돌려주는 함수 vars()와 함께 사용하는 예시
table = {k: str(v) for k, v in vars().items()}
table
>>> {'__name__': '__main__', '__builtin__': "<module 'builtins' (built-
     in)>", '__builtins__': "<module 'builtins' (built-in)>", ...

formattedkeys_list = [f'{k}: ' + '{' + k + '};' for k in table.keys()]
# 중괄호로 둘러싸인 k를 만드는 이유:
# **를 이용해 키워드 인자로 딕셔너리를 전달했을 때,
# k(key이름)에 대응되는 값(value)들이 중괄호 안에 치환된다.
formattedkeys_list
>>> ['__name__: {__name__};', '__builtin__: {__builtin__};',
     '__builtins__: {__builtins__};', '_ih: {_ih};', '_oh: {_oh};
     ', ... , '__file__: {__file__};']

message = " ".join(formattedkeys_list)
message
>>> '__name__: {__name__}; __builtin__: {__builtin__};
     __builtins__: {__builtins__}; _ih: {_ih}; _oh: {_oh};
     ...  __file__: {__file__};'

# message에서 join메소드를 통해 리스트 요소들을 묶어 문자열로 만들어주지 않으면
# .format(**table)은 사용할 수 없음
print(message.format(**table))'
>>> __name__: __main__;
    __builtin__: <module 'builtins' (built-in)>;
    __builtins__: <module 'builtins' (built-in)>;
    _ih: ['', "runcell(0, 'C:/Users/user/Downloads/untitled29.py')",
          ...
--------------------------
for x in range(1, 11):
    print('{0:2d} {1:3d} {2:4d}'.format(x, x*x, x*x*x))
>>>  1   1    1
     2   4    8
     ...
     9  81  729
    10 100 1000  # 2d, 3d, 4d: 숫자들이 각각 2,3,4자리를 차지하며 출력되도록
\end{minted}
\subsection{Manual String Formatting}
몇 가지 새로운 메소드들을 이용해서 수동으로 문자열을 포매팅 할 수도 있다.\\방금 전과 똑같은 제곱수, 세제곱수 표를 다른 방법으로 다시 포맷팅하면:
\begin{minted}
[
frame=lines,
framesep=3mm,
baselinestretch=1.3,
bgcolor=LightGray,
fontsize=\normalsize,
linenos
]
{python}
for x in range(1, 11):
    print(repr(x).rjust(2), repr(x*x).rjust(3), repr(x*x*x).rjust(4))
>>>  1   1    1
     2   4    8
     ...
     9  81  729
    10 100 1000

# .rjust()메소드를 사용하기 위해서는 숫자 값을 문자열로 변환시켜야 함
# 위 코드에서는 repr() 함수를 이용해서 변환함
for x in range(1, 11):
    print(x.rjust(2), (x*x).rjust(3),end=' ')
    print((x*x*x).rjust(4))
>>> AttributeError: 'int' object has no attribute 'rjust'
\end{minted}
\renewcommand{\arraystretch}{1.5}
\begin{table}[htb]
\centering
%\caption{A list of house plants and their details.}
\begin{tabular}{l|l}
\hline
\textbf{메소드}  & \textbf{연산}          \\
\hline\hline
str.rjust()      & 문자열을 오른쪽으로 줄맞춤(왼쪽을 공백으로 채워서)    \\
\hline
str.ljust()      & 문자열을 왼쪽으로 줄맞춤                            \\
\hline
str.center()     & 문자열을 가운데에 줄맞춤                            \\
\hline
str.zfill()      & 숫자 문자열의 왼쪽을 0으로 채움                      \\
\hline
\end{tabular}
\end{table}

\subsection{Old string formatting}
\%\ 연산자를 포맷팅에 사용하면 주어진 열 너비에 문자열을 채우는, 즉 문자열 보간을 효율적으로 수행할 수 있다.
\begin{minted}
[
frame=lines,
framesep=3mm,
baselinestretch=1.3,
bgcolor=LightGray,
fontsize=\normalsize,
linenos
]
{python}
# 10칸의 문자열 공간에서 오른쪽 정렬
print("^^^%10s@@@" % "hello!");
>>> ^^^    hello!@@@ # ^^^과 @@@사이에 10칸의 공간이 있음

# 10칸의 문자열 공간에서 왼쪽 정렬
print("^^^%-10s@@@" % "hi!");
>>> ^^^hello!    @@@

# 오른쪽 정렬하고 빈 공간은 0으로 채우기
print("^^^%010d@@@" % 12);
>>> ^^^0000000012@@@

# 소수점 아래 4자리까지 나타내고 왼쪽 정렬
print("^^^%-10.4f@@@" % 12.1);
>>> ^^^12.1000   @@@

# 오른쪽 정렬 후 부호(+) 표기하고 남는 공간 0으로 채우기
print("^^^%+010.4f@@@" % 12.1);
>>> ^^^+0012.1234@@@
\end{minted}
\section{Reading and Writing Files}
open()은 파일 객체를 돌려주는 함수이며 주로 \\
open('파일명', '모드', encoding = 'utf-8')와 같은 방식으로 호출된다. \\
첫 번째 인자에는 파일의 이름이 문자열로 담겨 있다. \\
두 번째 인자인 모드로는 파일이 어떤 방식으로 사용될지 가리키는 문자열이 들어간다.:\\
\(\rightarrow\) 'r': 읽기 전용 / 'a': 덧붙이기 / 'w': 쓰기 전용(같은 이름의 파일이 이미 존재한다면 이전에 있던 파일이 삭제됨) / '+r': 읽고 쓰기 . \\
모드는 선택적인 항목으로, 생략할 경우 'r'모드로 간주된다.
\begin{tcolorbox}[colback=white,colframe=black!40!white,title=참고: 파일 열기 모드]
\textbf{r+, w+, a+}: 읽기모드+쓰기모드. w+와 a+의 차이는 w와 a의 차이와 동일하다. 단, w+에서는 파일이 존재하지 않으면 새로 생성하지만 r+에서는 오류가 발생한다.
\textbf{rb, wb, ab, rb+, wb+, ab+}: Binary 포맷으로 읽거나 쓰도록 하는 모드이다. \\(포맷을 제외하면 위와 동일하다)
\end{tcolorbox}


with 키워드란?\\
어떤 파일을 입력, 출력할 때 반드시 open과 close를 해줘야한다.\\
이것을 하나의 with문으로 나타낼 수 있는데, 그렇게 되면 깜빡 잊고 파일을 close해주지 않는 실수를 방지할 수 있다. \\
기본적인 사용법은 \uline{with expression as target:}이다
\begin{minted}
[
frame=lines,
framesep=3mm,
baselinestretch=1.3,
bgcolor=LightGray,
fontsize=\normalsize,
linenos
]
{python}
file_data = open('file.txt', 'r', encoding="utf-8")
print(file_data.readline(), end="")  # 실행할 함수 예시
file_data.close()
\end{minted}
with 키워드를 반드시 사용해야 하는 것은 아니다. \\
하지만 위의 코드에서 close()를 호출하지 않는다면, 프로그램이 성공적으로 종료되었다 하더라도 file\_data.write()의 인자가 디스크에 완전히 기록되지 않을 수 있다.\\
따라서 파일을 이렇게 여는 대신, with키워드를 사용하면
\begin{minted}
[
frame=lines,
framesep=3mm,
baselinestretch=1.3,
bgcolor=LightGray,
fontsize=\normalsize,
linenos
]
{python}
with open('file.txt', 'r', encoding="utf-8") as file_data:
    # 함수는 동일하게 with문 안에 사용하면 됨
    print(file_data.readline(), end="")
\end{minted}
with문을 빠져나올 때 file\_data .close()를 자동으로 호출해준다.

\subsection{Methods of File Objects}
지금부터의 예제들은 f라는 파일 객체가 이미 만들어졌다고 가정한다.\\
(코드에서 with open( example.txt ) as f: 생략)

\paragraph{write, writelines}
두 메소드 모두 자동으로 줄넘김 해주는 기능은 없기 때문에\\
내용을 다음 줄에 적고 싶다면 개행문자를 직접 넣어주어야 한다.
\begin{minted}
[
frame=lines,
framesep=3mm,
baselinestretch=1.3,
bgcolor=LightGray,
fontsize=\normalsize,
linenos
]
{python}
# 1. write
f.write('hello Python\n')
f.write('googbye\n')

# 2. writelines : 리스트 안의 요소들을 한꺼번에 작성해줌
f.writelines(['word1', 'word2', 'line3\n', 'line4\n'])
\end{minted}
+ 만약 write하려는 값이 다른 자료형이라면, 입력하기 전 먼저 문자열 또는 바이트 객체로 변환해야 한다:
\begin{minted}
[
frame=lines,
framesep=3mm,
baselinestretch=1.3,
bgcolor=LightGray,
fontsize=\normalsize,
linenos
]
{python}
value = ('the answer', 42)
s = str(value)  # 튜플 -> 스트링
f.write(s)
\end{minted}

\paragraph{read}
f.read(size) - 파일에서 size 바이트만큼의 데이터를 읽는다.\\
size로 음수가 입력되거나 생략되면 파일의 \uline{텍스트 전체}를 읽어온다.
\begin{minted}
[
frame=lines,
framesep=3mm,
baselinestretch=1.3,
bgcolor=LightGray,
fontsize=\normalsize,
linenos
]
{python}
read_result = f.read()
print(read_result)
>>> hello Python
    googbye
    word1word2line3
    line4
# readline이나 readlines에서처럼 각각의 분리된 줄들을 출력하는 것 같이 보이지만
print(repr(read_result))
>>> 'hello Python\ngoogbye\nword1word2line3\nline4\n'
# 텍스트 전체를 읽어왔다는 것을 알 수 있음
\end{minted}

\paragraph{readline}
개행(줄넘김)문자인 '\textbackslash n' 을 기준으로 line을 구분해서 읽는다.\\
한 줄씩 읽어오기 때문에 대부분 반복문과 함께 쓰인다.
\begin{minted}
[
frame=lines,
framesep=3mm,
baselinestretch=1.3,
bgcolor=LightGray,
fontsize=\normalsize,
linenos
]
{python}
while True:
    line = f.readline()
    if not line: # 파일 읽기가 종료된 경우
        break
    print(line.strip())
>>> hello Python
    googbye
    word1word2line3
    line4
#(문장들 사이가 한 줄씩 띄워지는 것을 방지하기 위해
# .strip()으로 각각의 line들 끝에 붙어있는 \n을 제거해줌)
\end{minted}

\paragraph{readlines}
파일의 모든 줄을 읽어온다. 하지만 텍스트를 통째로 반환하는것이 아니라, 개행문자를 포함한 line들의 list 형태로 변환하여 돌려준다.
\begin{minted}
[
frame=lines,
framesep=3mm,
baselinestretch=1.3,
bgcolor=LightGray,
fontsize=\normalsize,
linenos
]
{python}
lines = f.readlines()
print(lines)
>>> ['hello Python\n', 'googbye\n', 'word1word2line3\n', 'line4\n']
for line in lines:
   print(line.split())
>>> hello Python
    googbye
    word1word2line3
    line4
\end{minted}
\paragraph{list(f)}
읽어온 문자열들을 포함하는 새로운 객체를 만드는 대신에 파일 객체 자체를
looping해서 line들을 읽는 것도 가능하다. 이렇게 하면 빠르고 간단하며 메모리 면에서도 효율적이다. \\
한 줄씩 출력하지 않고 readlines의 리스트처럼 확인하고 싶다면 list(f)를 출력하면 된다.
\vspace{-0.5cm}
\begin{minted}
[
frame=lines,
framesep=3mm,
baselinestretch=1.3,
bgcolor=LightGray,
fontsize=\normalsize,
linenos
]
{python}
for line in f:
        print(line, end='')
>>> hello Python
    googbye
    word1word2line3
    line4
print(f)
>>> <_io.TextIOWrapper name='example.txt' mode='r' encoding='utf-8'>
print(list(f))
>>> ['hello Python\n', 'googbye\n', 'word1word2line3\n', 'line4\n']
\end{minted}

\paragraph{tell, seek}
tell 메소드는 현재 커서의 위치를 반환해준다.\\
(바이너리 모드에서는 파일의 처음으로부터 계산한 바이트 수를 돌려주지만, 텍스트 모드에서는 숫자를 반환받아도 위치에 대한 정보를 바로 얻기는 어렵다.) \\
seek는 입력한 숫자에 해당하는 위치로 커서를 이동시킨다.
f.seek(offset, whence) \\형태로 사용하며, 위치는 기준점(whence)에 offset 을 더해서 계산한다.  \\
whence 값이 0 : 파일의 처음부터 측정 / 1: 현재 파일 위치 / 2 : 파일의 끝을 기준점으로.  \\
whence 은 생략 가능한데 그럴 경우 기본값인 0(파일의 처음)을 기준점으로 한다.
\begin{minted}
[
frame=lines,
framesep=3mm,
baselinestretch=1.3,
bgcolor=LightGray,
fontsize=\normalsize,
linenos
]
{python}
f = open('example.txt', 'rb+') # 모드: binary 포맷, 읽기+쓰기
f.write(b'0123456789abcdef')
>>> 16  # 출력된 문자의 개수 반환

f.seek(5)  # 첫번째 바이트 + 5 = 6 --> 6번째 바이트로 이동
>>> 5

f.read(1) # 현재 위치를 기준으로 1바이트
>>> b'5'

f.seek(-3, 2)  # 끝(whence = 2)에서부터 3번째 바이트로 이동
>>> 13

f.read(1)
>>> b'd'
\end{minted}