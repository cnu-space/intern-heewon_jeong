\section{Operating System Interface}
os 모듈은 운영체제와 상호작용하기 위한 함수들을 제공한다

os.getcwd() - 현재 작업 디렉토리 가져오기\\
os.chdir() - 작업 디렉토리 변경하기\\
os.listdir() - 디렉토리 내의 파일 및 서브 디렉토리 리스트\\
os.path.join() - 경로와 파일명 결합, 분할된 경로를 하나로 정리\\
os.path.dirnam() - 실행 파일의 상대경로\\
os.path.abspat() - 실행 파일의 절대경로\\
os.walk() - 시작 디렉토리부터 하위에 있는 모든 디렉토리들을 차례로 방문\\
os.path.splitext() - 파일 경로를 받아 파일 이름과 확장자로 분리하기\\
\newline
특정 디렉토리에서 시작해서 그 하위(하위 디렉터리 포함)의 모든 파일 중 파이썬 파일 \\(확장자 .py)만 출력해주는 함수를 만들 수 있다.\\
이 코드에서는 C:/Users/user 디렉터리에 있는 모든 파이썬 파일을 출력하도록 호출했다.
\begin{minted}
[
frame=lines,
framesep=3mm,
baselinestretch=1.3,
bgcolor=LightGray,
fontsize=\normalsize,
linenos
]
{python}
import os

def search(dirname):
    for (root, dir, files) in os.walk(dirname):
    # root: 최상위 디렉토리/ dir: 서브 디렉토리/ files: 파일
        for filename in files:
            ext = os.path.splitext(filename)[-1]
            if ext == '.py':
                print('%s/%s'%(path, filename))

search("c:/Users/user")
>>> c:/Users/user\PycharmProjects\pythonProject\.venv\Lib\site-packages\numpy\_typing/__init__.py
    c:/Users/user\PycharmProjects\pythonProject\.venv\Lib\site-packages\numpy\_utils/_convertions.py
    ...
\end{minted}
\section{File Wildcards}
glob 모듈을 사용하여 파일 와일드카드 검색을 구현할 수 있다. \\
glob모듈은 파일 시스템에서 패턴 매칭을 통해 파일 목록을 가져오는 기능을 제공하고, \\
와일드카드 문자로는 * 또는 ? 등이 사용된다.
(* : 0개 이상의 문자,  ? : 단일 문자 의미)
\begin{minted}
[
frame=lines,
framesep=3mm,
baselinestretch=1.3,
bgcolor=LightGray,
fontsize=\normalsize,
linenos
]
{python}
import glob

# 현재 디렉토리에서 모든 .txt 파일 검색
txt = glob.glob("*.txt")

# 현재 디렉토리에서 파일 이름이 'data'로 시작하고 .txt로 끝나는 파일 검색
data_txt = glob.glob("data*.txt")

# 현재 디렉토리에서 확장자가 .py 또는 .txt인 파일 검색
py_or_txt = glob.glob("*.{py,txt}")

# 하위 디렉토리까지 검색 (재귀적으로)
all = glob.glob("**/*", recursive=True)

# python1.txt, python2.txt, pythonA.txt, ..등과 매칭
python_x = glob.glob('python?.txt')

# HA0.py 부터 HA9.py까지 매칭
Homework = glob.glob('HA[0-9].py')

# 하위 디렉토리 포함하여 모든 .py 파일 찾기
# ** : 여러 디렉토리 레벨을 가로지르는 와일드카드
# recursive: 디렉토리가 아닌 파일이 나올때까지 재귀적으로 파일 찾기
all_py = glob.glob('**/*.py', recursive=True)
\end{minted}
\section{Command Line Arguments}
명령행 인자란 파이썬 스크립트를 실행할 때 프로그램에 전달하는 값들을 말한다. \\
스크립트 실행 시 프로그램 이름 뒤에 공백으로 구분하여 추가적으로 입력해주면 된다. (프로그램 이름은 항상 sys.argv[0]이므로, sys.argv[1]부터 실제 전달된 명령행 인자들이 저장된다.)
\begin{minted}
[
frame=lines,
framesep=3mm,
baselinestretch=1.3,
bgcolor=LightGray,
fontsize=\normalsize,
linenos
]
{python}
import sys
import os

try:  # 명령행 인자로 hello를 입력했을 때만 '안녕하세요'를 출력하도록
    if sys.argv[1] == 'hello':
        print('안녕하세요')
except:
    print('종료')

os.system('python main.py hello') # 터미널 언어-시스템 명령어 실행
>>> 안녕하세요
\end{minted}
+) argparse 모듈: 명령행 인자를 파싱하고 관리하는 데 편리한 기능을 제공한다.\\ (*파싱? - 받은 인자들을 잘게 나누어 원하는 형식으로 해석해서 사용하는 것\\
e.g.1)  두 숫자 입력받고 정수로 변환하여 더하기 연산
\begin{minted}
[
frame=lines,
framesep=3mm,
baselinestretch=1.3,
bgcolor=LightGray,
fontsize=\normalsize,
linenos
]
{python}
import argparse

# parser 만들기
parser = argparse.ArgumentParser(
    # 프로그램 이름 지정/도움말 출력 시 python add.py대신 add으로 보이도록
    prog = 'add',
    # -h 또는 --help 사용 시 출력되는 도움말
    description = "Add two numbers")

# 인수 추가하기
# -- 가 붙으면 키워드 인자를 뜻함.
# type = int : 정수형으로 자동 파싱됨
parser.add_argument('--x', type = int, help = '첫번째 수')
parser.add_argument('--y', type = int, help = '두 번째 수')
#-------------------------------------------
# (또는,)
# --를 붙이지 않으면 위치 인자로도 사용 가능
# required = True : 필수 인자 -> 입력하지 않으면 오류 발생
parser.add_argument('x', type = int, required = True, help = '첫번째 수')
parser.add_argument('y', type = int, required = True, help = '두 번째 수')
#-------------------------------
# 명령행 인자들을 파싱해서 args객체에 저장함
args = parser.parse_args()
print(f"Sum: {args.x + args.y}")

$ python add.py --x 10 --y 20
>>> Sum: 30
\end{minted}
e.g.2) 여러 개의 텍스트 파일을 입력받고, 각 파일의 처음 몇 줄만 출력하기
\begin{minted}
[
frame=lines,
framesep=3mm,
baselinestretch=1.3,
bgcolor=LightGray,
fontsize=\normalsize,
linenos
]
{python}
import argparse

parser = argparse.ArgumentParser(
    prog = 'top',
    description = 'Show top lines from each file')
# filenames: 위치 매개변수
# nargs='+': filenames는 한 개 이상의 인수를 받아야 함
parser.add_argument('filenames', nargs = '+')
# lines: 키워드 매개변수/ 아무 값도 입력되지 않으면 default값(10)사용
parser.add_argument('--lines', type = int, default = 10)
args = parser.parse_args()
print(args)

$ python top.py --lines=5 alpha.txt beta.txt
>>> Namespace(filenames=['alpha.txt', 'beta.txt'], lines=5)
'''
위의 명령행을 실행하면 args에 아래와 같이 저장된다:
args = Namespace(
    lines = 5,
    filenmaes = ['alpha.txt', 'beta.txt'])
# Namespace: 클래스 / args: Namespace의 인스턴스
'''
\end{minted}
\section{String Pattern Matching}
\begin{minted}
[
frame=lines,
framesep=3mm,
baselinestretch=1.3,
bgcolor=LightGray,
fontsize=\normalsize,
linenos
]
{python}
import re
'''
re.search(pattern, string):
문자열에서 패턴이 일치하는 첫 번째 위치를 찾아줌. (없으면 None 반환)
'''

# [0-9]: 숫자 한 개 이상 (2025, 7, 11, 등등..)
# \s*: 공백 문자(\s) 한 개 이상(*)
#     -> 공백 개수 상관 없이 매칭 가능(e.g.'2025년\n7월\t11일')
pattern = "([0-9]+년)\s*([0-9]+월)\s*([0-9]+일)"
date = re.search(pattern, "오늘 날짜: 2025년 7월 11일 / 내일 날짜: 2025년 7월 12일")

# 매칭 결과
date
>>> <re.Match object; span=(7, 20), match='2025년 7월 11일'>

# 인식한 패턴의 시작, 끝 인덱스
date.start()
>>> 7
date.end()
>>> 20

# 인식한 문자열
date.group()
>>> '2025년 7월 11일'

# 부분적으로 인식한 패턴 일치 결과값
date.groups()
>>> ('2025년', '7월', '11일')

'''
re.match(pattern, string): 문자열의 시작 부분에서만 매칭함
'''
re.match("c", "cdef")
>>> <re.Match object; span=(0, 1), match='c'>

re.match("c", "abcdef")  # 아무것도 반횐되지 않음
re.search("c", "abcdef")  # search는 시작 부분이 아니어도 인식 가능
>>> <re.Match object; span=(2, 3), match='c'>

'''
re.split(pattern, string): 패턴 기준으로 분할하기
'''
pattern = "[0-9]"
# 숫자들 (7, 1, 2)를 기준으로 쪼개짐
re.split(pattern, "내일은 7월 12일입니다.")
>>> ['오늘은 ', '월 ', '', '일입니다.'] # 리스트로 반환된다.

'''
re.findall(pattern, string): 패턴에 맞는 부분을 모두 찾아서 리스트로 돌려준다.
'''
re.findall('\d', "오늘 날짜: 2025년 7월 11일 / 내일 날짜: 2025년 7월 12일")
>>> ['2', '0', '2', '5', '7', '1', '1', '2', '0', '2', '5', '7', '1', '2']
re.findall('\d+', "오늘 날짜: 2025년 7월 11일 / 내일 날짜: 2025년 7월 12일")
>>> ['2025', '7', '11', '2025', '7', '12']
\end{minted}

\section{Mathematics}
math 모듈: 조금 더 복잡한 연산이 필요할 때 /
\begin{minted}
[
frame=lines,
framesep=3mm,
baselinestretch=1.3,
bgcolor=LightGray,
fontsize=\normalsize,
linenos
]
{python}
import math

# 삼각함수
angle = math.radians(30)

math.sin(angle)
>>> 0.49999999999999994
math.cos(angle)
>>> 0.8660254037844387
math.tan(angle)
>>> 0.5773502691896257

# 로그 함수
math.log(10) # 자연로그
>>> 2.302585092994046
math.log10(1000) # 밑이 10인 로그
>>> 3.0
math.log(100, 5)
>>> 2.8613531161467867

# 지수 함수
math.exp(100)
>>> 2.6881171418161356e+43

# 상수
math.pi  # π
>>> 3.141592653589793
math.e  # 자연 상수
>>> 2.718281828459045
math.tau  # 2π
>>> 6.283185307179586
\end{minted}
random 모듈: 난수 생성 기능
\begin{minted}
[
frame=lines,
framesep=3mm,
baselinestretch=1.3,
bgcolor=LightGray,
fontsize=\normalsize,
linenos
]
{python}
import random
list = ['A', 'B', 'C', 'D']

random.choice(list)
>>> 'B'

random.shuffle(list)
list
>>> ['D', 'A', 'B', 'C']

random.sample(range(100), 10)  # 0-99 중에서 표본 10개 추출
>>> [28, 21, 43, 9, 2, 6, 32, 62, 24, 35]

random.random()  # 0.0이상 1.0미만의 임의의 float값 반환
>>> 0.2829534807592088

random.randrange(1,7)   # 정수 1,2,...6 중 임의로 한 개 선택
>>> 4
\end{minted}
statistics 모듈: 숫자 데이터의 통계 계산을 위한 함수
\begin{minted}
[
frame=lines,
framesep=3mm,
baselinestretch=1.3,
bgcolor=LightGray,
fontsize=\normalsize,
linenos
]
{python}
import statistics
data = [1, 2, 3, 3, 3, 4, 4, 4, 5, 5]

# 산술 평균
statistics.mean(data)
>>> 3.4

# 기하 평균
statistics.geometric_mean(data)
>>> 3.116386929860295

# 중앙값
# +) median_high/low : 높은/낮은 중앙값
statistics.median(data)
>>> 3.5
statistics.median_high(data)
>>> 4
statistics.median_high(data)
>>> 3

# 최빈값 리스트
statistics.multimode(data)
>>> [3, 4]
\end{minted}
\section{Internet Access}
파이썬에서 인터넷에 접근하거나 요청을 보낼 수 있다.
\begin{minted}
[
frame=lines,
framesep=3mm,
baselinestretch=1.3,
bgcolor=LightGray,
fontsize=\normalsize,
linenos
]
{python}
from urllib.request import urlopen

url = 'http://python.com'
response = urlopen(url)

# read(): 데이터를 바이트로 읽어옴
# decode(): 바이트를 문자열로 변환
html = response.read().decode('utf-8')

# 처음 100글자 출력
print(html[:100])

# 처음 100줄 출력
lines = html.splitlines()
for line in lines[:100]:
    print(line)
\end{minted}
\section{Dates and Times}
파이썬은 실제 날짜, 시간을 나타내는 객체와 객체들을 이용한 연산, 형식 변환 등을 \\지원한다.
\begin{minted}
[
frame=lines,
framesep=3mm,
baselinestretch=1.3,
bgcolor=LightGray,
fontsize=\normalsize,
linenos
]
{python}
from datetime import date

now = date.today()
now
>>> datetime.date(2025, 7, 11)

now.strftime("%m-%d-%y. %d %b %Y is a %A on the %d day of %B.")
>>> '07-11-25. 11 Jul 2025 is a Friday on the 11 day of July.'

birthday = date(2005, 4, 19)
age = now - birthday
age.days
>>> 7388

next_birthday = birthday.replace(year = 2026)
Dday = next_birthday - now
Dday.days
>>> 282
\end{minted}