%5.1
\vspace{-1cm}
\section{More on Lists}

The list data type has some more methods. Here are all of the methods of list objects:
\renewcommand{\arraystretch}{1.5}
\begin{table}[htb]
\centering
%\caption{A list of house plants and their details.}
\begin{tabular}{|l|l|l|}
\textbf{Method} & \textbf{Argument}      & \textbf{Operation}                                                                                       \\
\hline\hline
append          & x (item)               & Add an item to the end of the list                                                                       \\
\hline
extend          & iterable               & Extend the list by appending all the items from the iterable                                             \\
                &                        & (similar to A[len(A):] = iterable)                                                                       \\
\hline
insert          & i(index), x(item)      & Insert an item at a given position                                                                       \\
\hline
remove          & item                   & Remove the first item \uline{whose value is equal to x}.                                                 \\
                &                        & (\textit{ValueError} if there is no such item).                                                          \\
\hline
pop             & index                  & Remove the item \uline{at the given position}, and return it                                             \\
\hline
clear           &                        & Remove all items from the list                                                                           \\
\hline
index           & x(, i\_start, i\_end)  & Return index of the first item whose value is x.                                                         \\
                &                        & (ValueError if there is no such item)                                                                    \\
\hline
count           & item                   & Return the number of times x appears in the list                                                         \\
\hline
sort            & *, key=, reverse=      & Sort the items of the list in place (key=비교 기준)                                                       \\
\hline
reverse         &                        & Reverse the elements of the list in place                                                                \\
\hline
copy            &                        & Return a shallow copy of the list. (Similar to A[:])
\end{tabular}
\label{tbl:example-table}
\end{table}

\newpage

\begin{minted}
[
frame=lines,
framesep=3mm,
baselinestretch=1.3,
bgcolor=LightGray,
fontsize=\normalsize,
linenos
]
{python}
fruits = ['orange', 'apple', 'pear', 'banana', 'kiwi', 'apple', 'banana']

fruits.count('apple')  #0
fruits.count('tangerine')  #0

fruits.index('banana')  #3
fruits.index('banana', 4)
# kiwi(=fruits[4]) 이후에 나타나는 banana의 인덱스를 돌려줌

fruits.reverse()
# ['banana', 'apple', 'kiwi', 'banana', 'pear', 'apple', 'orange']
fruits.sort(reverse=True)
# ['pear', 'orange', 'kiwi', 'banana', 'banana', 'apple', 'apple']

fruits.pop() # apple -> 인덱스 값이 정해지지 않으면 마지막 요소를 삭제하고 돌려줌

fruits.sort(key = lambda x : x[2])
# ['pear', 'orange', 'grape', 'banana', 'banana', 'apple', 'apple', 'kiwi']
# 각 아이템의 세번째 글자(인덱스 2)를 기준으로 정렬
\end{minted}

If statement can be followed by zero or more elif parts, and else part is optional
We use elif short for ‘else if’ in order to avoid excessive indentation

* when we’re comparing the same value to several constants, or checking for specific types or attributes, match statement would useful (see 2.7.)

\newpage

%=================================================================================
\subsection{Using Lists as Stacks}
stacks: "Last-in, first-out"
-> 아이템 추가: append(),  삭제: pop()
\begin{minted}
[
frame=lines,
framesep=3mm,
baselinestretch=1.3,
bgcolor=LightGray,
fontsize=\normalsize,
linenos
]
{python}
stack = [3, 4, 5]

stack.append(6) # [3, 4, 5, 6]
stack.append(7) # [3, 4, 5, 6, 7]

stack.pop() # 7 : 가장 최근에 추가했던 아이템을 삭제하고 돌려줌
stack.pop() # 6
\end{minted}

\subsection{Using Lists as Queues}
queues: “first-in, first-out”
lists are not efficient for this purpose. 리스트의 끝에서 append와 pop을 하는 것은 빠르지만, 리스트의 시작 부분에서 insert와 pop을 하는 것은 느림. (맨 앞에 요소가 추가되거나 제거될 때 나머지 요소들도 한 칸씩 움직여야 하기 때문에)

그 대신, 더 효율적으로 큐를 구현하기 위해서는 collections 모듈의 deque를 사용한다.
deque(데크): double-ended queue
list와 매우 유사하고 사용할 수 있는 메서드도 대부분 일치하지만, 앞과 뒤에서 데이터를 처리 가능한 양방향 자료형임. 추가로 다음과 같은 메서드가 제공됨\\
- appendleft(x): 데크의 왼쪽에 x 추가\\
  : 리스트에서 insert(0, x)와 같음
- popleft(x): 데크의 왼쪽에서 요소를 제거
  : 리스트에서 pop(0)과 같음

\subsection{List Comprehensions}
- 다른 시퀀스 또는 iterable의 요소들에 대해 어떤 연산을 했을 때의 결과값을 요소로 가지는 새로운 리스트 만들 때.
- 또는 그 요소들 중 어떠한 조건을 만족시키는 데이터만을 요소로 가지는 부분 시퀀스를 만들 때 유용함
list = [(expression of x) for x in (sequence)]

똑같은 리스트를 세 가지 방법으로 만들어볼 수 있음\\
예제: 0부터 5까지의 정수 n에 대해 $n^2$을 요소로 가지는 리스트 만들기\\
-> [0, 1, 4, 9, 16, 25]\\
\newline
1. for 루프로 빈 리스트에 요소 채우기
\begin{minted}
[
frame=lines,
framesep=3mm,
baselinestretch=1.3,
bgcolor=LightGray,
fontsize=\normalsize,
linenos
]
{python}
squares = []
for x in range(6):
    squares.append(x**2)
\end{minted}
2. map(), lambda expression
\begin{minted}
[
frame=lines,
framesep=3mm,
baselinestretch=1.3,
bgcolor=LightGray,
fontsize=\normalsize,
linenos
]
{python}
squares = list(map(lambda x: x**2, range(6)))
\end{minted}
3. 리스트 내포
\begin{minted}
[
frame=lines,
framesep=3mm,
baselinestretch=1.3,
bgcolor=LightGray,
fontsize=\normalsize,
linenos
]
{python}
squares = [x**2 for x in range(6)]
\end{minted}
\newpage
for, if 문을 포함하는 리스트 내포\\
\newline
e.g.1) 순서쌍 리스트 만들기
\begin{minted}
[
frame=lines,
framesep=3mm,
baselinestretch=1.3,
bgcolor=LightGray,
fontsize=\normalsize,
linenos
]
{python}
# 1부터 5까지의 서로 다른 두 수
a=[(y, x) for x in range(1,6) for y in range(1,6) if y!=x]
print(a)
print(len(a))
>>> [(1, 2), (1, 3), (1, 4), (1, 5), ... (5, 1), (5, 2), (5, 3), (5, 4)]
    20

# 중복을 허용하지 않을 때
a=[(x, y) for x in range(1,6) for y in range(1,6) if y<x]
'''
(or, [(x, y) for x in range(1,6) for y in range(1,x)]
-> y가 x보다 작은 값만 가지도록 범위를 설정할 수도 있음)
'''
print(a)
print(len(a))
>>> [(2, 1), (3, 1), (3, 2), ... (5, 1), (5, 2), (5, 3), (5, 4)]
    10  # 순서쌍의 개수는 절반이 됨
\end{minted}

* 튜플을 요소로 하는 경우 꼭 괄호로 감싸야 함\\
a=[x, y for x in ... ]\\
\(\rightarrow\) SyntaxError: did you forget parentheses around the comprehension target?\\
\newpage
e.g.2) 집합 기본 연산
\begin{minted}
[
frame=lines,
framesep=3mm,
baselinestretch=1.3,
bgcolor=LightGray,
fontsize=\normalsize,
linenos
]
{python}
A=[-1,-2,3,-4,-5,6]
B=[4,5,6,7,8]
C=[-x for x in A]
# create a new list from A with the value multiplied by -1

dif=[x for x in A if not x in B]  # 차집합 A-B
# filter the list to exclude numbers in B

inter=[x for x in C if x in B]  # 교집합 B, C
# filter the list to only include numbers both in B and C

union=list(set([abs(x) for x in A+B])) # 합집합 A, B
# apply a function (absolute) to all the elements
\end{minted}

복잡한 표현식이나 중첩 함수를 포함한 리스트 내포도 가능함. \\(* 중첩 함수: 함수 안에 다른 함수가 정의된 형태)
\begin{minted}
[
frame=lines,
framesep=3mm,
baselinestretch=1.3,
bgcolor=LightGray,
fontsize=\normalsize,
linenos
]
{python}
# 0 부터 2π 까지 π/2씩 증가하는 라디안 값들을 도 단위로 바꾸기
# 결과값 요소의 소수점 이하는 생략하고 문자열로 변환

from math import *
# 사용한 메소드: floor, degrees, pi
import numpy as np
# 사용한 메소드: arange, 구간의 크기를 정수가 아닌 값으로 설정하기 위해

list_deg= [str(floor(degrees(i))) for i in np.arange(0, pi*2.1, pi/2)]

>>> ['0', '90', '180', '270', '360']
\end{minted}
\newpage
주의:\\
1) 조건부 표현식(삼항연산자) 문법: else문에서 변수에 값 또는 표현식 할당할 수 없음\begin{minted}
[
frame=lines,
framesep=3mm,
baselinestretch=1.3,
bgcolor=LightGray,
fontsize=\normalsize,
linenos
]
{python}
A=[x if x%2 == 1 else 0 for x in range(1,10)] # 값
>>> [1, 0, 3, 0, 5, 0, 7, 0, 9]

A=[x if x%2 == 1 else x*2 for x in range(1,10)] # 표현식
>>> [1, 4, 3, 8, 5, 12, 7, 16, 9]

A=[x if x%2 == 1 else x=0 for x in range(1,10)]
# SyntaxError: cannot assign to conditional expression
\end{minted}
%
2) if-else 문은 for문 앞에서 사용해야 하고, \\
반대로 for문과 if문만을 쓸 때는 if문이 for문 뒤에 와야 함.
\begin{minted}
[
frame=lines,
framesep=3mm,
baselinestretch=1.3,
bgcolor=LightGray,
fontsize=\normalsize,
linenos
]
{python}
odd_only=[x for x in range(10) if x%2 == 1]
>>> [1,3,5,7,9]

odd_only=[x if x%2 == 1 for x in range(10)]
# SyntaxError: expected 'else' after 'if' expression

#-------------------------

find_odd=[x if x%2 == 0 else 'odd' for x in range(10)]
>>> [0, 'odd', 2, 'odd', 4, 'odd', 6, 'odd', 8, 'odd']

find_odd=[x for x in range(10) if x%2 == 0 else 'odd']
# SyntaxError: invalid syntax
\end{minted}
\newpage
3) 이중 for문을 리스트 내포로 표현 시 바깥쪽 for문의 변수를 먼저 정의해야함.
\begin{minted}
[
frame=lines,
framesep=3mm,
baselinestretch=1.3,
bgcolor=LightGray,
fontsize=\normalsize,
linenos
]
{python}
vec = [[1,2,3], [4,5,6], [7,8,9]]

p=[num for elem in vec for num in elem]
>>> [1, 2, 3, 4, 5, 6, 7, 8, 9]

q=[num for num in elem for elem in vec]
# NameError: name is not defined
\end{minted}

\subsection{Nested(중첩) List Comprehensions}
: 리스트 내포의 첫 번째 표현식 자리에 또다른 listcomp를 넣을 수도 있음.

 e.g.) 3x4 행렬
\begin{minted}
[
frame=lines,
framesep=3mm,
baselinestretch=1.3,
bgcolor=LightGray,
fontsize=\normalsize,
linenos
]
{python}
# 3X4: 길이 4인 리스트 3개
matrix = [
    [1, 2, 3, 4],
    [5, 6, 7, 8],
    [9, 10, 11, 12],
]

# 전치행렬 만들기
T = [[row[i] for row in matrix] for i in range(4)]
>>> [ [1, 5, 9], [2, 6, 10], [3, 7, 11], [4, 8, 12] ]
# (matrix안에 있는 리스트들을 각각 A, B, C라 하면,)
# T = [ [A[0], B[0], C[0]], ... ,[A[3], B[3], C[3]] ]
\end{minted}
\newpage
마찬가지로 for 반복문을 이용해서 똑같이 만들 수 있음
\begin{minted}
[
frame=lines,
framesep=3mm,
baselinestretch=1.3,
bgcolor=LightGray,
fontsize=\normalsize,
linenos
]
{python}
'''for문'''
T = []
for i in range(4):
    T.append([row[i] for row in matrix])
# inner list comprehension is evaluated -
# in the context of the for clause that follows it

'''이중 for문'''
T = []
for i in range(4):
    transposed_row = [] # 전치행렬의 행이 될 리스트
    for row in matrix:
        transposed_row.append(row[i])
   T.append(transposed_row)
# i값에 대한 for문이 가장 바깥쪽에 있으므로
# .append(row[i])를 matrix안의 모든 row들에 대해 완료해야 다음 i값으로 넘어갈 수 있음
\end{minted}

복잡한 구문들보다는 내장 함수를 사용하는 것이 좋음. \\
e.g. zip( )
\begin{minted}
[
frame=lines,
framesep=3mm,
baselinestretch=1.3,
bgcolor=LightGray,
fontsize=\normalsize,
linenos
]
{python}
A = [1, 2, 3, 4],
B = [5, 6, 7, 8],
C = [9, 10, 11, 12]
matrix = [A, B, C]

list(zip(*matrix)) # 2.9 unpacking argument lists
# = list(zip(A, B, C))
>>> [[1, 5, 9], [2, 6, 10], [3, 7, 11], [4, 8, 12]]

# zip 함수 ?
x = [1, 2, 3]
y = [4, 5, 6]
list(zip(x, y))
>>> [(1, 4), (2, 5), (3, 6)]
\end{minted}


\section{The del statement}
값 대신 인덱스를 사용해서 리스트 항목 삭제하기\\
pop() 메소드 또한 요소를 제거하기 위해 인덱스를 입력받지만, 삭제된 값을 돌려준다는 점에서 del 문과 다름
\begin{minted}
[
frame=lines,
framesep=3mm,
baselinestretch=1.3,
bgcolor=LightGray,
fontsize=\normalsize,
linenos
]
{python}
a = [-1, 1, 66.25, 333, 333, 1234.5]

del a[0] # 첫 번째 아이템 삭제
>>> [1, 66.25, 333, 333, 1234.5]

del a[2:4] # 슬라이스 삭제
>>> [1, 66.25, 1234.5]

del a[:] # 리스트 비우기
>>> []

del a # 리스트가 할당된 변수 자체를 삭제함
# 이후에 다시 a를 참조하면 에러가 발생함
\end{minted}

\section{Tuples and Sequences}

시퀀스 자료형: string, list, range, \uline{tuple}\\
튜플은 쉽표로 분리된 여러 개의 값들로 구성됨.\begin{minted}
[
frame=lines,
framesep=3mm,
baselinestretch=1.3,
bgcolor=LightGray,
fontsize=\normalsize,
linenos
]
{python}
# 튜플 요소에는 인덱스를 이용해 접근할수있음
t = 12345, 54321, 'hello!'
t[0]
>>> 12345

# 튜플은 중첩될 수 있음
# 중첩 튜플, 또는 다른 표현식의 일부로 튜플을 사용하려면 괄호로 둘러싸야 함
u = t, (1, 2, 3, 4, 5)
>>> ((12345, 54321, 'hello!'), (1, 2, 3, 4, 5))

# 튜플은 불변 객체
t[0] = 88888

# 가변 객체를 포함하는 것은 가능
v = ([1, 2, 3], [3, 2, 1])
\end{minted}
튜플은 리스트와 비슷해 보이지만 다른 상황에서 다른 목적으로 사용된다. \renewcommand{\arraystretch}{1.5}
\begin{table}[htb]
\centering
%\caption{A list of house plants and their details.}
\begin{tabular}{l|l|l}
                       & \textbf{List}            & \textbf{Tuple}        \\
\hline\hline
가변성                  & mutable                  & immutable             \\
\hline
구성요소들의 데이터 타입  & homogeneous              & heterogeneous         \\
\hline
요소 접근 방식           & iterating over the list  & unpacking or indexing *\\
\end{tabular}
\end{table}
\\ * namedtuple의 경우 속성(attribute)을 통해 요소에 접근할 수도 있다. \\
(다음 페이지 예시 참고)
\begin{minted}
[
frame=lines,
framesep=3mm,
baselinestretch=1.3,
bgcolor=LightGray,
fontsize=\normalsize,
linenos
]
{python}
from collections import namedtuple

data = [
     ('스폰지밥', 12, 15.34),
     ('징징이', 21, 25.4),
     ('집게사장', 56, 17.78),
]

Employee = namedtuple('Employee', 'name, age, height')
# Employee는 name, age, height 속성을 가지는 클래스처럼 작동함
# (namedtuple은 'name, age, height'을 자동으로 .split()해줌)

data=[Employee(emp[0], emp[1], emp[2]) for emp in data]
# _make 함수 사용 -> [Employee._make(emp) for emp in data]
print(data)
>>> [ Employee(name='스폰지밥', age=12, height=15.34),
      Employee(name='징징이', age=21, height=25.4),
      Employee(name='집게사장', age=29, height=17.78) ]

emp = data[1] # 2번째 직원
emp.name # 징징이
emp.age # 21
emp.height # 25.4

Emp_dict=[emp._asdict() for emp in data)] # _asdict 함수: 사전으로 변환
print(Emp_dict)
>>> [{'name':'스폰지밥', 'age':12, 'height'=15.34},
     {'name':'징징이', 'age':21, 'height'=25.4},
     {'name':'집게사장', 'age':29, 'height'=17.78}]
\end{minted}
\newpage
빈 튜플 또는 한 개의 아이템만 가지는 튜플
\begin{minted}
[
frame=lines,
framesep=3mm,
baselinestretch=1.3,
bgcolor=LightGray,
fontsize=\normalsize,
linenos
]
{python}
empty = () # 빈 튜플은 비어있는 괄호 쌍으로 만듦
par = ('hello') # 괄호만 -> 문자열
com = 'hello', # 쉼표만(또는 괄호+쉼표) -> 튜플

empty
len(empty)
type(empty)
>>> ()
    0
    <class 'tuple'>
par
len(par)
type(par)
>>> hello
    5
    <class 'str'>
com
len(com)
type(com)
>>> ('hello',)
    1
    <class 'tuple'>
\end{minted}
\newpage
튜플 패킹, 언패킹:
\begin{minted}
[
frame=lines,
framesep=3mm,
baselinestretch=1.3,
bgcolor=LightGray,
fontsize=\normalsize,
linenos
]
{python}
t = 12345, 54321, 'hello!'  # 튜플 패킹
x, y, z = t  # 시퀀스 언패킹

x, y, z = 12345, 54321, 'hello!'  # 다중 할당 (multiple assignment)
# 단순히 튜플 패킹과 시퀀스 언패킹의 조합임
\end{minted}
시퀀스 언패킹에서는 우변에 튜플 이외에도  어떤 시퀀스든 올 수 있음. 단, 좌변에는 \\
우변의 시퀀스에 있는 요소들과 같은 개수의 변수들이 있어야 한다.
\begin{minted}
[
frame=lines,
framesep=3mm,
baselinestretch=1.3,
bgcolor=LightGray,
fontsize=\normalsize,
linenos
]
{python}
# 언패킹 가능한 형태라면 어떤 꼴이든 상관 없음
l,m,n = [1, 2, 3]
i,j,k = range(3)
x,y,z = "abc"

print(x,y,z)
>>> a b c
\end{minted}

\section{Sets}
Set(집합)란 순서가 없고 중복되는 요소가 없는 자료구조이다. 주로 어떤 요소가 데이터 컬렉션 안에 존재하는지 검사(멤버십 테스팅)할 때, 또는 중복되는 항목을 제거하기 위한 목적으로 사용된다. 세트 객체는 여러가지 집합 연산도 지원한다.

세트는 요소들을 중괄호로 감싸거나 set() 함수를 사용해 만든다. 이때 빈 세트는 둘 중 set()함수를 사용해야 한다. (\{ \}는 세트가 아닌 빈 딕셔너리를 의미함)
\begin{minted}
[
frame=lines,
framesep=3mm,
baselinestretch=1.3,
bgcolor=LightGray,
fontsize=\normalsize,
linenos
]
{python}
basket = {'apple', 'orange', 'apple', 'pear', 'orange', 'banana'}
print(basket)
>>> {'orange', 'banana', 'pear', 'apple'}  # 중복 제거됨
'peach' in basket   # 멤버십 테스팅
>>> False

a = set('AaBbCDAB')
b = set('1a2bCDz34')

b  # unique letters in b
>>> {'4', 'b', '2', 'D', 'C', 'a', '1', 'z', '3'}  # 순서 무작위
len(b)
>>> 9

a-b   # 차집합
>>> {'B', 'A'}
a|b   # 합집합 ('|' = 'or')
>>> {'4', 'b', 'B', 'D', '2', 'C', 'a', '1', 'z', '3', 'A'}
a&b   # 교집합
>>> {'a', 'b', 'D', 'C'}
# 대칭 차집합 구하는 여러 가지 방법
a^b
>>> {'4', 'B', '2', '1', 'z', '3', 'A'}
(a-b)|(b-a)
>>> {'4', '1', 'B', 'z', '3', '2', 'A'}
(a|b) - (a&b)
>>> {'4', 'B', '2', '1', 'z', '3', 'A'}

a == b  # a와 b가 완전히 같은 집합인지
>>> False
a.isdisjoint(b) # a와 b가 완전히 다른 집합인지 (교집합 없음)
>>> False
\end{minted}

집합은 가변 객체이기 때문에 리스트와 비슷한 여러 함수가 지원된다.
\renewcommand{\arraystretch}{1.5}
\begin{table}[htb]
\centering
\begin{tabular}{|l|l|l|}
\hline
\textbf{Method} & \textbf{Argument}      & \textbf{Operation}                       \\
\hline\hline
add             & item         & Add an item to the set                             \\
\hline
update          & iterable     & Extend by adding all the items from the iterable   \\
\hline
remove          & item         & Remove the item                                    \\
                &              & (자료 안에 값이 없으면 ValueError발생)                \\
\hline
discard         & item         & Safely remove the item                              \\
                &              & (있으면 삭제/없으면 아무 일도 일어나지 않음)            \\
\hline
pop             &              & Remove random item and return it                    \\
\hline
clear           &              & 모든 요소 제거 , 제거 후 print하면 set()가 출력됨       \\
\hline
\end{tabular}
\end{table}
%(집합의 맨 앞에 있는 요소가 제거되는 것처럼 보이지만 집합 내부에서 정해진 순서가 없기 때문에 결국은 임의의 요소가 삭제된다고 할 수 있다
\\리스트 내포와 같이, 집합에서도 컴프리헨션을 사용할 수 있다:
\begin{minted}
[
frame=lines,
framesep=3mm,
baselinestretch=1.3,
bgcolor=LightGray,
fontsize=\normalsize,
linenos
]
{python}
a = {x for x in 'abracadabra' if x not in 'abc'}
>>> {'r', 'd'}

num=[n for n in range(2,36)]
s={(x,y) for x in num for y in num if x<=y and x*y==36}
>>> {(3, 12), (4, 9), (6, 6), (2, 18)}
\end{minted}
set.add() 함수를 이용해 for문으로도 만들 수 있다.
\begin{minted}
[
frame=lines,
framesep=3mm,
baselinestretch=1.3,
bgcolor=LightGray,
fontsize=\normalsize,
linenos
]
{python}
s = set()
num=[n for n in range(2,36)]
for x in num:
    for y in num:
        if x<=y and x*y==36:
            s.add((x,y))  # s.add(x,y):오류 발생(2 argument given)
print(s)
\end{minted}
\newpage
주의: 리스트와 달리 세트는 불변 객체만을 요소로 가질 수 있기 때문에\\
표현식 자리에 리스트, 딕셔너리, 또다른 세트 등 가변 객체가 오면 오류가 발생한다.\\
\(\rightarrow\) 세트 안에 세트의 특징을 가지는 객체를 포함하고 싶다면 frozenset을 쓸 수 있다\\
frozenset는 set와 거의 동일한 기능을 하지만, 튜플과 같이 한 번 할당된 값은 변경이 불가능한 객체이다.
\begin{minted}
[
frame=lines,
framesep=3mm,
baselinestretch=1.3,
bgcolor=LightGray,
fontsize=\normalsize,
linenos
]
{python}
num=[n for n in range(2,36)]

s={{x,y,z} for x in num for y in num for z in num
      if x!=y and y!=z and x!=z and x*y*z==36}
# TypeError: unhashable type 'set'

s_tup={(x,y,z) for x in num for y in num for z in num
      if x!=y and y!=z and x!=z and x*y*z==36}
>>> {(3, 2, 6), (6, 2, 3), (2, 3, 6), (2, 6, 3), (6, 3, 2), (3, 6, 2)}

s_fro={frozenset((x,y,z)) for x in num for y in num for z in num
      if x!=y and y!=z and x!=z and x*y*z==36}
>>> {frozenset({2, 3, 6})}
    # 순서만 다른 숫자쌍들은 전부 중복으로 인식되어 하나만 남기고 삭제됨
\end{minted}




\section{Dictionaries}

지금까지 시퀀스 타입 list, tuple, range, str, 세트 타입 set, frozenset등을 살펴보았다.\\
사전은 또다른 자료형인 mapping type이다.\\
숫자로 인덱싱되는 시퀀스형과 달리 사전은 'key'로 인덱싱된다. \\
이떄 key로는 불변형 자료를 사용한다.
\newpage
사전 안에서 키는 중복될 수 없다. 만약 이미 사용중인 키로 어떤 값을 저장하면, 그 키와 함께 저장되었던 원래 값은 새로운 값으로 대체된다. \\존재하지 않는 키로 값을 추출하려고 하면 오류가 발생한다.\\
사전에서 사용되고 있는 모든 키의 리스트를 돌려주는 함수:\\
- list(dict) 삽입 순서대로  \\
- sorted(dict) 정렬된 순서대로
\begin{minted}
[
frame=lines,
framesep=3mm,
baselinestretch=1.3,
bgcolor=LightGray,
fontsize=\normalsize,
linenos
]
{python}
bd = {'정희원': '0419', '장혜원': '0515', '최정원': '0306'}

bd['최정원'] = '0307' # 해당 키의 원래 값이 새로운 값으로 대체됨
# {'정희원': '0419', '장혜원': '0515', '최정원': '0307'}

del bd['정희원'] # '정희원': '0419' 삭제
bd['박하은'] = '1213'
# {'장혜원': '0515', '최정원': '0307', '박하은': '1213'}

list(bd) # 삽입 순서
# ['장혜원', '최정원', '박하은']
sorted(bd) # 정렬
# ['박하은', '장혜원', '최정원']

'정희원' not in bd
# True
bd['정선경'] # 존재하지 않는 키의 값 추출
>>> KeyError: '정선경'
\end{minted}
\newpage
key-value쌍들이 시퀀스로 묶여 있을때 dict()생성자를 이용하여 바로 사전으로 만들 수 있다.
\begin{minted}
[
frame=lines,
framesep=3mm,
baselinestretch=1.3,
bgcolor=LightGray,
fontsize=\normalsize,
linenos
]
{python}
s={('apple', 'toamto'), ('pear', 'carrot')}

l=[((1,2), ('A','B')),    # 키-값 쌍1
   ('hello', [456,789])]  # 키-값 쌍2
   # 가변 객체인 리스트는 값 자리에만 올 수 있음

dict()
>>> {} #빈 사전

dict(s) # 시퀀스형이 아닌 set형 자료에도 dict를 적용할 수 있음
>>> {'pear': 'carrot', 'apple': 'toamto'}

dict(l)
>>> {(1, 2): ('A', 'B'), 'hello': [456, 789]}
\end{minted}
컴프리헨션을 이용해서 임의의 키와 값 표현식을 가지는 사전을 만들 수도 있다
\begin{minted}
[
frame=lines,
framesep=3mm,
baselinestretch=1.3,
bgcolor=LightGray,
fontsize=\normalsize,
linenos
]
{python}
num=[4,5,6]
str='ABC'

{(x,x+1): y for x in num for y in str}
>>> {(4, 5): 'C', (5, 6): 'C', (6, 7): 'C'}
# (4, 5): 'A', (4, 5): 'B' 등등 같은 키 값을 가지는 쌍들은 모두 덮어씌워져서
# 값 자리에는 'C'만 남게 됨

{(num[i],num[i]+1):str[i] for i in range(3)}
>>> {(4, 5): 'A', (5, 6): 'B', (6, 7): 'C'}
\end{minted}
특히 키가 간단한 문자열일 때는, 키워드 인자((매개변수)=(값))들을 사용하면 키-값 쌍을 지정하기 더 쉽다.
\begin{minted}
[
frame=lines,
framesep=3mm,
baselinestretch=1.3,
bgcolor=LightGray,
fontsize=\normalsize,
linenos
]
{python}
dict(Alice= 'pliot', Bob= 'astronaut')
>>> {'Alice': 'pliot', 'Bob': 'astronaut'}

# 따옴표로 둘러싸인 값 또는 숫자를 매개변수로 사용하면 오류 발생
dict('Alice' = 'pliot', 300000='km/s')
>>> SyntaxError: expression cannot contain assignment, perhaps you meant "=="?
\end{minted}
딕셔너리에서도 마찬가지로 여러 가지 함수를 사용할 수 있다
\renewcommand{\arraystretch}{1.5}
\begin{table}[htb]
\centering
%\caption{A list of house plants and their details.}
\begin{tabular}{|l|l|l|}
\hline
\textbf{Method} & \textbf{Argument} & \textbf{Operation}                            \\
\hline\hline
setdefault      & key, value     & 키와 값을 추가함                                   \\
\hline
update          & 키워드 인자     & 이미 사용중인 key의 값을 수정, 없다면 새로 추가       \\
                & (key=value)    & * 변수명 형태의 키가 아닐 경우 {키:값}(사전형)으로 입력  \\
\hline
pop             & key 또는       & 키와 값을 꺼내고 값을 반환                          \\
                & key, default   & 존재하지 않는 key일 경우 dafault를 반환함            \\
\hline
popitem         &                & 마지막 아이템 제거                                  \\
\hline
clear           &                & 모든 키-값 쌍 삭제                                    \\
\hline
\end{tabular}
\end{table}
\\예시:
\begin{minted}
[
frame=lines,
framesep=3mm,
baselinestretch=1.3,
bgcolor=LightGray,
fontsize=\normalsize,
linenos
]
{python}
d={}

d.setdefault('a') # 딕셔너리에 '키' 추가
d.setdefault('b', 100) # 키, 값 추가
d
>>> {'a': None, 'b': 100}

d.update(a=50)
d.update({1:100}) # 숫자 key -> '딕셔너리'형태로 입력해야 함
d.update(zip(['b'],[1000]))
d
>>> {'a': 50, 'b': 1000, 1: 100}

d.pop('b')
>>> 1000
d.pop('c',0)
>>> 0  # 존재하지 않는 키 이므로 기본값 반환

d.popitem()  # 마지막 키-값 쌍 삭제
>>> (1, 100)
\end{minted}

\section{Looping Techniques}
사전에서 items()는 키-값 쌍을, keys(), values()는 키, 값 각각을 돌려준다.
\begin{minted}
[
frame=lines,
framesep=3mm,
baselinestretch=1.3,
bgcolor=LightGray,
fontsize=\normalsize,
linenos
]
{python}
fruits={'apple':'사과', 'grape':'포도'}

for eng, kor in fruits.items():
    print(f'{eng} - {kor}')
>>> apple - 사과
    grape - 포도

key_value=list(zip(fruits.keys(), fruits.values()))
>>> [('apple', '사과'), ('grape', '포도')]
item=list(fruits.items())
key_value == item
>>> True
\end{minted}
\newpage
enumerate()는 어떤 값과 그 값의 위치를 함께 반환해주는 함수이다.
\begin{minted}
[
frame=lines,
framesep=3mm,
baselinestretch=1.3,
bgcolor=LightGray,
fontsize=\normalsize,
linenos
]
{python}
# e.g.1) 오류 메시지와 오류 발생 위치 함께 출력하기
logs_example = [
"INFO: User logged in",
"ERROR: Not authorized user",
"INFO: Data processed",
"INFO: Data processed",
"ERROR: File not found",
]
for line, log in enumerate(logs_example):
    if "ERROR" in log:
        print(f"Error found at line {line+1}: {log}")

>>> Error found at line 2: ERROR: Not authorized user
    Error found at line 5: ERROR: File not found
#-------------------------------------------
# e.g.2) 수열의 원소와 항 번호 함께 출력하기
fib=[0, 1, 1, 2, 3, 5, 8, 13, 21, 34, 55, 89, 144, 233, 377, 610, 987, 1597]

for index, value in enumerate(fib, start=1): # 시작 인덱스 1로 설정
    if index%5==0 or index==len(fib):
    # 숫자 5개마다 줄넘김/리스트의 마지막 숫자 뒤에는 쉼표 생략
        split='\n'
    else:
        split=', '
    print(f'a{index} = {value}',end=split)

>>> a1 = 0, a2 = 1, a3 = 1, a4 = 2, a5 = 3
    a6 = 5, a7 = 8, a8 = 13, a9 = 21, a10 = 34
    a11 = 55, a12 = 89, a13 = 144, a14 = 233, a15 = 377
    a16 = 610, a17 = 987, a18 = 1597
\end{minted}
zip()함수를 사용하면 여러 개의 시퀀스들을 한 번에 병렬적으로 돌면서 처리할 수 있다.
\vspace{-1cm}
\begin{minted}
[
frame=lines,
framesep=3mm,
baselinestretch=1.3,
bgcolor=LightGray,
fontsize=\normalsize,
linenos
]
{python}
for number in '123':
    for upper in 'ABC':
        for lower in 'abc':
           print(number, upper, lower)
>>> 1 A a
    1 A b
    1 A c
    1 B a
    1 B b
    ...
    3 C a
    3 C b
    3 C c

for number, upper, lower in zip('123', 'ABC', 'abc'):
    print(number, upper, lower)
>>> 1 A a
    2 B b
    3 C c
#-------------------------------------------
# 61p 딕셔너리 예제
num=[4,5,6]
str='ABC'

{(x,x+1): y for x in num for y in str}
>>> {(4, 5): 'C', (5, 6): 'C', (6, 7): 'C'}

# zip함수 사용
{(x,x+1): y for x, y in zip(num,str)}
>>> {(4, 5): 'A', (5, 6): 'B', (6, 7): 'C'}
\end{minted}
reversed()는 시퀀스를 거꾸로 반복하는 함수이다.\\
* 주의: reversed와 reverse의 차이점\\
reversed(시퀀스)는 원래의 시퀀스에는 영향을 주지 않고 역순으로 순회할 수 있도록 반복자를 반환해준다. 반면 시퀀스.reverse()는 기존 시퀀스 안의 항목들을 재배치 시키기 때문에 순서가 보존되어야 하는 시퀀스일 경우 사용에 주의해야 한다.
\begin{minted}
[
frame=lines,
framesep=3mm,
baselinestretch=1.3,
bgcolor=LightGray,
fontsize=\normalsize,
linenos
]
{python}
# 회문 검사 함수
def palindrom(w):
    w_reversed=''.join(reversed(w))
    print(w,'-->',w_reversed,end=' | ')
    if w == w_reversed:
        print('회문입니다')
    else:
        print('회문이 아닙니다')

palindrom('hello')
palindrom('대학생학대')
>>> hello --> olleh | 회문이 아닙니다
    대학생학대 --> 대학생학대 | 회문입니다
\end{minted}
반대로 sorted() 함수가 반환해주는 반복자는 정렬된 순서로 항목에 접근한다.
\begin{minted}
[
frame=lines,
framesep=3mm,
baselinestretch=1.3,
bgcolor=LightGray,
fontsize=\normalsize,
linenos
]
{python}
basket= ['apple', 'orange', 'apple', 'orange', 'banana']
for f in sorted(basket):
    print(f)
>>> apple
    apple
    banana
    orange
    orange

# 중복되는 항목 없이 정렬된 시퀀스를 얻기 위해 sorted()와 set()조합이 자주 사용됨

unique_sorted_basket=[fruit for fruit in sorted(set(basket))]
>>> ['apple', 'banana', 'orange']
\end{minted}

\section{More on Conditions}

\textbf{comparison operators}\\
in, not in은 값이 컨테이너 안에 있는지 검사하고,\\
is, is not은 두 객체가 완전히 같은지 검사한다.\\
비교연산자들끼리 우선순위는 모두 동등하고, 산술 연산자들보다는 항상 낮다.

\textbf{and, or, not}\\
우선순위: not > and > or \\
\(\rightarrow\) e.g.) A and not B or C는 (A and (not B)) or C와 같다. \\
and와 or는 \uline{단락 회로 연산자}로도 불린다.\\
왼쪽에서 오른쪽 순서로 값이 구해지다가 결과가 결정되자마자 중단된다.\\
예를 들어, 표현식 A 참, B 거짓, C 참이라면
\uline{A and B and C}는 B에서 값 구하기를 멈추고 표현식 C의 값은 구하지 않는다.

논리값이 아닌 일반적인 값일 경우, 항상 마지막으로 구해진 값이 반환된다.
\begin{minted}
[
frame=lines,
framesep=3mm,
baselinestretch=1.3,
bgcolor=LightGray,
fontsize=\normalsize,
linenos
]
{python}
''' and 연산자 '''
a = 1 and 2
print(a) >>> 2  # 1은 참, 다음 값도 확인 -> 마지막으로 확인한 값은 2
a = 6 and 0 and 3
print(a) >>> 0  # 0은 거짓으로 간주됨, 더 이상 확인하지 않고 0을 반환

''' or 연산자 '''
a = 0 or 7
print(a) >>> 7  # 0은 거짓이므로 다음 값 확인: 7
a = 9 or 7
print(a) >>> 9  # 9가 참이므로 더 이상 확인하지 않고 9 반환
\end{minted}

비교 또는 다른 논리 표현식의 결과를 변수에 대입하는 것도 가능함
\begin{minted}
[
frame=lines,
framesep=3mm,
baselinestretch=1.3,
bgcolor=LightGray,
fontsize=\normalsize,
linenos
]
{python}
obj1, obj2, obj3 = '', 'python', 'hello'
non_null = obj1 or obj2 or obj3
non_null >>> 'python' # 빈 문자열: 거짓
''' None, False,  0, 0.0, 0j (복소수) 또는 빈 시퀀스, 컨테이너('', [], {}, set(), range(0)) 등의 값들은 모두 거짓으로 간주함'''
\end{minted}
\textbf{walrus operator :=}
\\(변수) := (표현식) 형태로 사용함. 표현식의 결과를 변수에 할당하고, 동시에 반환한다.
\begin{minted}
[
frame=lines,
framesep=3mm,
baselinestretch=1.3,
bgcolor=LightGray,
fontsize=\normalsize,
linenos
]
{python}
# 입력받은 단어의 길이가 5 이상이면 단어와 단어의 길이 출력
str=input('단어 입력: ')
length=len(str)
if length >= 5:
    print(f'문자열: {str}, 길이: {length}')
else:
    print('너무 짧음')
#---------------------------------
# := 연산자 사용
if (length:=len(str:=input('단어 입력: ')))>=5:
    print(f'문자열: {str}, 길이: {length}')
else:
    print('너무 짧음')
\end{minted}
\section{Comparing Sequences and Other Types}

\begin{minted}
[
frame=lines,
framesep=3mm,
baselinestretch=1.3,
bgcolor=LightGray,
fontsize=\normalsize,
linenos
]
{python}
[1, 2, 3]              < [1, 2, 4]
(1, 2, 3, 4)           < (1, 2, 4)
# 첫 번째 요소부터 순차적으로 비교

'999999999999'<'ABC' < 'C' < 'PYTHON' < 'Python'
# 대문자<소문자, '숫자'<'문자'

(1, 2)                 < (1, 2, -1)
# 하나가 다른 시퀀스의 첫부분 일부라면 둘중 짧은 시퀀스가 더 작음

(1, 2, 3)             == (1.0, 2.0, 3.0)
# 다른 타입이지만 비교 가능

(1, 2, ('aa', 'ab'))   < (1, 2, ('abc', 'a'), 4)
# ('aa', 'ab') < ('abc', 'a')
\end{minted}